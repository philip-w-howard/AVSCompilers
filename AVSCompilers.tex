%% Based on a TeXnicCenter-Template by Gyorgy SZEIDL.
%%%%%%%%%%%%%%%%%%%%%%%%%%%%%%%%%%%%%%%%%%%%%%%%%%%%%%%%%%%%%

%----------------------------------------------------------
%
\documentclass[letterpaper,12pt,openany,reqno]{book}%
%
%----------------------------------------------------------
% This is a sample document for the standard LaTeX Book Class
% Class options
%       --  Body text point size:
%                        10pt (default), 11pt, 12pt
%       --  Paper size:  letterpaper (8.5x11 inch, default)
%                        a4paper, a5paper, b5paper,
%                        legalpaper, executivepaper
%       --  Orientation (portrait is the default):
%                        landscape
%       --  Printside:   oneside, twoside (default)
%       --  Quality:     final(default), draft
%       --  Title page:  titlepage, notitlepage
%       --  Columns:     onecolumn (default), twocolumn
%       --  Start chapter on left:
%                        openright(no, default), openany
%       --  Equation numbering (equation numbers on right is the default):
%                        leqno
%       --  Displayed equations (centered is the default):
%                        fleqn (flush left)
%       --  Open bibliography style (closed bibliography is the default):
%                        openbib
% For instance the command
%          \documentclass[a4paper,12pt,reqno]{book}
% ensures that the paper size is a4, fonts are typeset at the size 12p
% and the equation numbers are on the right side.
%
\usepackage{amsmath}%
\usepackage{amsfonts}%
\usepackage{amssymb}%
\usepackage{graphicx}
%----------------------------------------------------------
\newtheorem{theorem}{Theorem}
\newtheorem{acknowledgement}[theorem]{Acknowledgement}
\newtheorem{algorithm}[theorem]{Algorithm}
\newtheorem{axiom}[theorem]{Axiom}
\newtheorem{case}[theorem]{Case}
\newtheorem{claim}[theorem]{Claim}
\newtheorem{conclusion}[theorem]{Conclusion}
\newtheorem{condition}[theorem]{Condition}
\newtheorem{conjecture}[theorem]{Conjecture}
\newtheorem{corollary}[theorem]{Corollary}
\newtheorem{criterion}[theorem]{Criterion}
\newtheorem{definition}[theorem]{Definition}
\newtheorem{example}[theorem]{Example}
\newtheorem{exercise}[theorem]{Exercise}
\newtheorem{lemma}[theorem]{Lemma}
\newtheorem{notation}[theorem]{Notation}
\newtheorem{problem}[theorem]{Problem}
\newtheorem{proposition}[theorem]{Proposition}
\newtheorem{remark}[theorem]{Remark}
\newtheorem{solution}[theorem]{Solution}
\newtheorem{summary}[theorem]{Summary}
\newenvironment{proof}[1][Proof]{\textbf{#1.} }{\ \rule{0.5em}{0.5em}}
%----------------------------------------------------------
\begin{document}

\frontmatter
\title{A Very Simple Compiler Book}
\author{Philip W. Howard}
\date{2017}
\maketitle
\tableofcontents

\chapter{Introduction}

Given the existence of other very good books suitable for a course on compilers, why did I choose to write another one? There are several reasons:
  
\begin{enumerate}
\item Most compilers books seem to be written from the perspective that the course is preparing students to actually write a production compiler. That is not the perspective of the compiler courses I've taught. While the students in my course write a complete compiler, I readily acknowledge that very few of my students will work on a compiler as part of their career. The value of the course has more to do with helping the students to become better developers than in preparing them for a particular field. I have found that this difference in perspective alters what material I find most suitable for the courses I teach. 
\item I find the price of most text books almost criminal. By writing my own text, I can make electronic copies freely available and make printed copies available for a reasonable price.
\end{enumerate}

\mainmatter

\part{Scanning}

\chapter{Regular Expressions}

This chapter will discuss regular expressions, what they are for, and how to use them.

\chapter{Creating a Scanner from Regular Expressions}

This chapter will cover the algorithms for converting a list of RE's to code that can process them

\section{Thompson's}

This section presents Thompson's construction.

\section{Subset Construction}

This section presents the Subset Construction

\chapter{Scanner Code}
\section{table driven}
Here is a table driven scanner
\section{switch statement}
Here is a table drive scanner

\chapter{Automatically Generated Scanners}
here is a description of flex.

\part{Parsing}
\chapter{Context Free Grammars}
\chapter {Top-down recursive-descent parsers}
\chapter {Bottom-up parsers}
\chapter {Automatically generated parsers}
Description of bison.

\subsection{Subsection}

This is just some text under a subsection.

\subsubsection{Subsubsection}

This is just some text under a subsubsection.

\paragraph{Subsubsubsection}

This is just some  text under a subsubsubsection.

\subparagraph{Subsubsubsubsection}

This is just some text under a subsubsubsubsection.


\begin{itemize}
\item Bullet item 1

\item Bullet item 2

\end{itemize}

\begin{description}
\item[Description List] Each description list item has a term followed by the
description of that term.

\item[Bunyip] Mythical beast of Australian Aboriginal legends.
\end{description}

\section{Theorem-Like Environments}

The following theorem-like environments (in alphabetical order) are available
in this style.

\begin{acknowledgement}
This is an acknowledgement
\end{acknowledgement}

\begin{algorithm}
This is an algorithm
\end{algorithm}

\begin{axiom}
This is an axiom
\end{axiom}

\begin{case}
This is a case
\end{case}

\begin{claim}
This is a claim
\end{claim}

\begin{conclusion}
This is a conclusion
\end{conclusion}

\begin{condition}
This is a condition
\end{condition}

\begin{conjecture}
This is a conjecture
\end{conjecture}

\begin{corollary}
This is a corollary
\end{corollary}

\begin{criterion}
This is a criterion
\end{criterion}

\begin{definition}
This is a definition
\end{definition}

\begin{example}
This is an example
\end{example}

\begin{exercise}
This is an exercise
\end{exercise}

\begin{lemma}
This is a lemma
\end{lemma}

\begin{proof}
This is the proof of the lemma.
\end{proof}

\begin{notation}
This is notation
\end{notation}

\begin{problem}
This is a problem
\end{problem}

\begin{proposition}
This is a proposition
\end{proposition}

\begin{remark}
This is a remark
\end{remark}

\begin{summary}
This is a summary
\end{summary}

\begin{theorem}
This is a theorem
\end{theorem}

\begin{proof}
[Proof of the Main Theorem]This is the proof.
\end{proof}

\appendix

\chapter{The First Appendix}

The \verb"\appendix" command should be used only once. Subsequent appendices can
be created using the Chapter command.

\chapter{The Second Appendix}

Some text for the second Appendix.

This text is a sample for a short bibliography. You can cite a book by making use of
the command \verb"\cite{KarelRektorys}": \cite{KarelRektorys}. Papers can be cited
similarly: \cite{Bertoti97}. If you want multiple citations to appear in a single set
of square brackets you must type all of the citation keys inside a single citation,
separating each with a comma. Here is an example: \cite{Bertoti97, Szeidl2001,
Carlson67}.

\begin{thebibliography}{9}
\bibitem {KarelRektorys}Rektorys, K., \textit{Variational methods in Mathematics,
Science and Engineering}, D. Reidel Publishing Company,
Dordrecht-Hollanf/Boston-U.S.A., 2th edition, 1975

\bibitem {Bertoti97} \textsc{Bert\'{o}ti, E.}:\ \textit{On mixed variational formulation
of linear elasticity using nonsymmetric stresses and displacements}, International
Journal for Numerical Methods in Engineering., \textbf{42}, (1997), 561-578.

\bibitem {Szeidl2001} \textsc{Szeidl, G.}:\ \textit{Boundary integral equations for
plane problems in terms of stress functions of order one}, Journal of Computational and
Applied Mechanics, \textbf{2}(2), (2001), 237-261.

\bibitem {Carlson67}  \textsc{Carlson D. E.}:\ \textit{On G\"{u}nther's stress functions
for couple stresses}, Quart. Appl. Math., \textbf{25}, (1967), 139-146.
\end{thebibliography}

\backmatter

\chapter{Afterword}

The back matter often includes one or more of an index, an afterword,
acknowledgements, a bibliography, a colophon, or any other similar item. In
the back matter, chapters do not produce a chapter number, but they are
entered in the table of contents. If you are not using anything in the back
matter, you can delete the back matter TeX field and everything that follows it.
\end{document}
